\documentclass{article}
\usepackage{amsmath,amssymb}
\usepackage{fullpage}
\usepackage[authoryear]{natbib}
\usepackage{graphicx}
\usepackage[hidelinks]{hyperref}
%\usepackage{latexml} % for \iflatexml
\usepackage{color}
\usepackage[dvipsnames]{xcolor}

\bibliographystyle{plainnat}


\title{Population structure along the genome}
\author{
Han Li
\\and\\
Peter Ralph
}


\begin{document}

Other phrases that might go into a better title?
\begin{itemize}
    \item How selection affects relatedness
    \item Inhomogeneous kinship
    \item There's more than one kinship matrix
    \item Local PCA reveals inversions and the effects of selection
    \item Unsupervised learning of genomic patterns of relatedness
\end{itemize}


%%%%%%%%%%%%
\section{Introduction}


Catchy start: the kinship matrix goes back to almost Mendel and is essential in GWAS;
however, it is well-known that actual relatednesses have a lot of noise about the expected value,
and depend on where on the genome you look;
this is why scans for selective sweeps work.

Review of kinship matrix: 
it's either an expected kinship, given the pedigree;
or an estimated genome-wide average.
Wright's path coefficients \citep{wright1943isolation}.
Why it helps with confounding for GWAS.
Graham \& Vince's paper, maybe.
Like IBD, is only well-defined in a known pedigree, up to the founders.
Kinship and confounding reviewed in \citet{astle2009population}.

Kinship matrices differ for sex chromosomes and the like.

Review of selection causing differential patterns along the genome.
Locally everything is treelike (gene trees);
kinship matrix is an average of these (write equation for this).
Selective sweeps cause local recent ancestry/short trees.
Balancing selection causes deep trees.
Background selection, shallow ones.
Extreme examples of free gene flow in some places between species: e.g. Heliconius.
Introgression may be nonuniform, e.g. neanderthal, others.
Refs: hitchhiking \citep{maynardsmith1974hitchhiking},
\citet{kim2011hitchhiking} study hitchhiking in a spatially subdivided population,
\citet{mcvean2007structure} looks at the effect of selection on LD.
\citet{barton2000genetic} reviews hitchhiking.
\citet{bierne2010distinctive} discusses how hitchhiking effect decreases with geographic distance.
\citet{charlesworth2003review} reviews patterns of diversity, relating spatial structure to effects of selection.

Review of methods looking along the genome:
argweaver, HMM between species, ???

What is "population structure"?
Asks which "populations" are closely related, more diverged, how much diversity do they harbor.
Often geographic.
Vital in exploratory data analysis.
It is a summary of kinship: 
lack of migration between pops causes a deficit in connections through the pedigree,
and so affects kinship.
Wright defined $F_{ST}$ in \citep{wright1949genetical}, and says ``It has probably occurred to the reader that the coefficientof inbreeding may mean very different
things in different cases.''

Review of methods for visualizing pop structure:
PCA, structure \citep{falush2003inference}, EEMS, \citep{petkova2014visualizing}, \citep{yang2012modelbased},
Maps of heterozygosity \citep{ramachandran2005support}.
Genealogical interpretation of PCA by \citet{mcvean2009genealogical}.
Other semi-related stuff:
estimation of covariance matrices;
local pca(?);

\subsection{Applications}

We apply this to three species.

humans:
effects of selection are seen \citep{sabeti2006positive}.
\citet{duforetfrebourg2015detecting} showed that variants known to underly locally adaptive traits correlated most strongly with PCs.
\citet{tian2008analysis} noted that removing the segregating inversion on chromosome 8 changed the PCA for a set of 1400 Europeans
dataset(s)

drosophila:
effects of selection \citep{sella2009pervasive},
inversions,
dataset(s)

medicago:


%%%%%%%%%%%%
\section{Methods}


%%%%%%%%%%%%
\section{Results}


%%%%%%%%%%%%
\section{Discussion}

%%%%%%%%%%%%%
\bibliography{references}

\end{document}
