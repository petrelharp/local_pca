\documentclass[smaller]{beamer}
\usepackage{color}
\usepackage{ulem}

\usepackage{amsmath}
\usepackage{amssymb}
\usepackage{color}

\renewcommand{\familydefault}{\sfdefault}

% syntactic coloring
% by default, beamer-colors are
%    normal text is black on white.
%    alerted text is red.
%    example text is a dark green (green with 50% black).
%    structure is set to a light version of MidnightBlue (more precisely, 20% red, 20% green, and 70% blue).

\setbeamercolor{key point}{parent=alerted text}

\def\struct{\usebeamercolor[fg]{structure}}
\def\key{\em \usebeamercolor[fg]{key point}}
\def\aside{\scriptsize \usebeamercolor[fg]{example text}}
\def\newthing{\usebeamercolor[fg]{key point}}
\def\oldthing{\usebeamercolor[fg]{structure}}
\def\defthing{\usebeamercolor[fg]{structure}}

% \definecolor{blue}{rgb}{0.05,0.6,1}
% \definecolor{orange}{rgb}{1,0.549,0}
% \definecolor{dark-red}{rgb}{.4,.1,.1}
% \definecolor{pink}{rgb}{1,.8,.8}
% \definecolor{dark-green}{rgb}{.1,.6,.1}
% \definecolor{light-green}{rgb}{.8,1,.8}
% \definecolor{blue-white}{rgb}{.95,.95,1}
% \definecolor{purple}{rgb}{1, 0.0, 0.6}

\newcommand{\R}{\mathbb{R}}
\newcommand{\Q}{\mathbb{Q}}
\newcommand{\Z}{\mathbb{Z}}
\newcommand{\N}{\mathbb{N}}
\newcommand{\C}{\mathbb{C}}

\renewcommand{\P}{\mathbb{P}}
\newcommand{\E}{\mathbb{E}}

\newcommand{\var}{\mathop{\mbox{Var}}}
\newcommand{\cov}{\mathop{\mbox{cov}}}
\newcommand{\median}{\mathop{\mbox{median}}}
%\newcommand{\det}{\mathop{\mbox{det}}}
\newcommand{\supp}{\mathop{\mbox{supp}}}
\newcommand{\sgn}{\mathop{\mbox{sgn}}}

\newcommand{\conv}{\mathop{\mbox{conv}}}
\newcommand{\deq}{\stackrel{\scriptscriptstyle{d}}{=}}
\newcommand{\dcv}{\stackrel{\scriptscriptstyle{d}}{\longrightarrow}}


\newcommand{\figcredit}[1]{{\begin{flushright}\usebeamercolor[fg]{structure} \it \tiny #1 \end{flushright}}}



\def\basedir{files}

\newcommand{\widebar}[1]{\overline{#1}}
\newcommand{\finger}{ \vtop{ \vskip-5pt \hbox{ \includegraphics[width=.68in]{finger_right}} } }

\renewcommand<>{\sout}[1]{
  \only#2{\beameroriginal{\sout}{#1}}
  \invisible#2{#1}
}
\setbeamertemplate{blocks}[default]
\usecolortheme{rose}

\mode<presentation>
{
  % \usetheme{default}
  \usetheme{boxes}
  % or ...
  \usefonttheme[options]{structuresmallcapsserif}

  \setbeamercovered{transparent}
  % or whatever (possibly just delete it)
}


\usepackage[english]{babel}
% or whatever

%% PLR COMMENTED THESE:
\usepackage[latin1]{inputenc}
% \usepackage{times}
% \usepackage[T1]{fontenc}
% Or whatever. Note that the encoding and the font should match. If T1
% does not look nice, try deleting the line with the fontenc.


\title % (optional, use only with long paper titles)
{ Using local PCA to summarize how relatedness varies along the genome }

\author % (optional, use only with lots of authors)
{Peter Ralph and Han Li}

\institute[UO]
{
    University of Oregon \& Institute of Ecology and Evolution \\
    \textit{and} University of Southern California
  }

\date % (optional)
{June 26, 2017}


\begin{document}


\begin{frame}
  \titlepage
\end{frame}



%%%%%%%%
\begin{frame}{Principal Components Analysis (PCA)}

    \includegraphics<1>[width=\textwidth]{\basedir/novembre-map-genes-mirror-geography-crop}

    \only<1>{\aside (Novembre et al 2008)}

    \includegraphics<2>[width=\textwidth]{\basedir/maize_pca}

    \only<2>{\aside (van Heerwaarden et al 2010)}
    % http://www.pnas.org/content/108/3/1088.full

\end{frame}

%%%%%%%%
\begin{frame}{\ldots describes the covariance matix}

    average over trees

    describes patterns of relatedness through averaging patterns in local trees

\end{frame}

%%%%%%%%
\begin{frame}{``Population structure''}

    describes how everyone is related

    but local adaptation or hybridization

        leads to selection against migrants

        and differential gene flow

    but background selection makes coalescence faster near constrained sites

    \ldots so what is it?

\end{frame}

%%%%%%%%
\begin{frame}{}


    Population structure
    is historical patterns of interbreeding, migration, and population sizes.

    \pause

    Linked selection distorts the resulting genealogical patterns.

\end{frame}


%%%%%%%%
\begin{frame}{lostruct}

    method figure

\end{frame}


%%%%%%%%
\begin{frame}{lostruct}

    an R package

    with shell scripts

    and templated Rmarkdown report

    shrink method figure

\end{frame}

%%%%%%%%
\begin{frame}{Simulation: local adaptation}

    environments: (hot, dry) -- (hot, wet) -- (cold, wet) 

    genes: (hot/cold) -- (wet/dry)

    25MB (0.625M) with 1000 evenly spaced loci with $s=\pm0.001$, 1000 diploids in each population and 1\% migration
     with simuPOP + msprime

\end{frame}

%%%%%%%%
\begin{frame}{Data: African \textit{D.~melanogaster} (DPGP)}

\end{frame}

%%%%%%%%
\begin{frame}{Data: European Humans}


\end{frame}


%%%%%%%%
\begin{frame}{Data: \textit{Medicago truncatula}}
\end{frame}


%%%%%%%%
\begin{frame}{Conclusions}

    there isn't always a single ``population structure''

    (more) evidence for widespread effects of linked selection

    lostruct is a visualization tool

    \url{https://github.com/petrelharp/local_pca}
    
\end{frame}

%%%%%%%%
\begin{frame}{Thanks}

    theory, coding, oregon

\end{frame}


%%%%%%%%
\begin{frame}{Robust?}

    mutation rate variation

    recombination rate variation (*)

    missing data (**)

\end{frame}

\end{document}
